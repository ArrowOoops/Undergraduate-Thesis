% !TeX root = ../main.tex

\chapter{总结和展望}

\section{全文总结}

本文研究一阶段基于候选框、特征融合机制以及注意力机制的目标检测器ASSD的整体结构与实现细节,考虑到原论文是基于VOC2007与2012数据集进行的验证,本文采用遥感图像数据集对ASSD的性能进行了测试,验证了其算法的有效性。

本文首先讨论了目标检测算法的背景和意义,同时介绍了国内外的研究现状。在相关技术介绍部分主要分析了了本实验中利用到的各项主要概念、技术与原理;包括卷积神经网络、Resnet网络、目标检测器的分类及特点、注意力机制以及遥感图像的介绍。之后介绍了其网络结构、候选框机制、特征融合机制与注意力单元,并分析了ASSD的整体结构与它的优势所在。然后在服务器上配置好实验环境并对遥感图像1-3米分辨率的数据集进行测试,根据目标检测算法常用的评估标准进行处理,最后得出实验结果并分析。


\section{未来展望}
考虑到遥感图像在军事领域上的重要作用,对于其上的目标检测需要有高效的性能。一阶段目标检测器的优势在于其分析速度快,虽然在检测结果的准确率方面比两阶段目标检测器效果差,但是满足了速度的需求。本实验中遥感图像数据集的图像个数较少,识别的物体类别数也不多,随着遥感技术的不断发展,未来遥感图像相关的数据集的数量与质量会更加丰富,这样对模型的训练会有很大的帮助。同时,将来可以在一些诸如金字塔结构的优化(调整各层的尺度)、主干网络的调整、候选框的选择(增加一些新的宽高比甚至增加倾斜的候选框)、学习率的调整等方面对ASSD进行改进,使之更加适应贴合遥感图像数据集。相信在深度学习与并行处理器飞速发展的当下,目标检测领域未来必然会有新的突破。