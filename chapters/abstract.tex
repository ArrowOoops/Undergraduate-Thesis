% !TeX root = ../main.tex

\ustcsetup{
  keywords = {
    目标检测,遥感图像,一阶段目标检测器,注意力机制,特征融合机制
  },
  keywords* = {
    Object Detection,Remote Sensing Image,One-Stage Detector,Attention Mechanism,Feature Fusion
  },
}

\begin{abstract}
随着深度学习与计算机视觉的飞速发展,目标检测成为近年来的一个热门研究方向。
目标检测算法分为传统检测方法与基于深度学习的检测方法,而基于深度学习的目标检测算法分为一阶段和两阶段的方法。
随着遥感技术的飞速发展,遥感图像的数量与质量得到了极大的提升。
目前,遥感已广泛应用于农业、林业、军事侦察及环境监测等领域,遥感正以其强大的生命力展现出广阔的发展及应用前景。
ASSD是一阶段使用候选框的目标检测器,它是基于SSD进行改进的,使用特征融合机制来丰富语义,使得小目标的识别更加准确,并引入了注意力机制来对提取到的特征图进行处理,对图片上感兴趣的区域进行着重分析。
本实验研究ASSD的各层结构,并测试其在遥感图像数据集上的检测效果。
\end{abstract}

\begin{abstract*}
With the rapid development of Deep Learning and Computer Vision, Object Detection has become a hot research direction.
Object Detection algorithm counts traditional detection methods and methods based on Deep Learning, while Object Detection based on Deep Learning is divided into one-stage and two-stage methods.
The quantity and quality of Remote Sensing Image have been greatly improved with the help of the development of Remote Sensing Technology.
At present, Remote Sensing Technology has been widely used in agriculture, forestry, military reconnaissance, environmental monitoring and other fields. With its strong vitality, Remote Sensing Technology is showing broad development and application prospects.
ASSD is the One-Stage Object Detector based on Anchor.It is improved from SSD,using Feature Fusion to enrich
the semantics which makes it have a greater detection accuracy on small scale objects.It also use Attention Mechanism to process Feature Map in order to focus on analysing the concerned region. 
This experiment study the structure of ASSD, and observe its detection effect on Remote Sensing Image.
\end{abstract*}
